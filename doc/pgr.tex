\documentclass[11pt,a4paper]{article}
\input{config}

\begin{document}
\titlepageandcontents

%---------------------------------------------------------------------------
\section{Zadání}
Cílem bylo implementovat globální osvětlování pomocí radiosity prostřednictvím platformy OpenCL.

Platforma OpenCL umožňuje akcelerovat obecné výpočty prostřednictvím paralelizace a využítí výpočetní síly grafického procesoru. Právě tímto směrem je zaměřen i tento projekt.

Důraz je kladen především na vhodnou paralelizaci problému (neboť to je hlavní výhoda možnosti využití grafického akcelerátoru) a tím i maximální zrychlení celého procesu osvětlování oproti seriové implementaci na procesoru.

Kvalita zobrazení a podoba výsledné scény není hlavní prioritou projektu. Stejně tak nebylo cílem vytvořit optimální algoritmus pro výpočet radiosity či dílčích prerekvizit, jako je dělení plošek scény či návrh scény samotné.

Součástí řešení je i implementace řešená stejným způsobem, pouze bez využití OpenCL, tedy sériově na procesoru. Tato implementace je využita k porovnání kvality výsledků i průběhu, ale především pro porovnání rychlosti a výsledného zrychlení.

%---------------------------------------------------------------------------
\section{Použité technologie}
Technologie použité pro projekt samotný jsou zobrazeny v následujícím přehledu. Všechny knihovny je třeba mít instalovány tak, aby bylo možno je volat standardním způsobem.
\begin{itemize}
  \item \textbf{OpenCL}. Žádné požadovky na minimální verzi, nutný je pouze podporovaný hardware a  odpovídající implementace. Pro překlad projektu je potřeba mít i hlavičkové soubory.
  \item \textbf{OpenGL}. Vykreslování probíhá pomocí standardního zobracovacího průběhu OpenGL. Pro překlad projektu je potřeba mít i hlavičkové soubory.
  \item \textbf{Knihovna SDL}. Práce s okny je zajištěna pomocí knihovny SDL. Tato knihovna zastřešuje práci s okny v systému napříč různými platformami a hardwarem. Pro překlad projektu je potřeba mít i hlavičkové soubory.
  \item \textbf{Knihovna PGR}. Tato knihovna zapouzdřuje práci s okny. Pochází z cvičení předmětu PGR 2013 na FIT VUT v Brně. Je přejata beze změny. K projektu je přiložena.
  \item \textbf{Knihovna GLEW}. Potřebná pro knihovnu PGR. Pro překlad projektu je potřeba mít i hlavičkové soubory.
  \item \textbf{Knihovna GLee}. Potřebná pro knihovnu PGR. K projektu je přiložena.
\end{itemize}

Pro překlad je potřeba standardních nástrojů \textbf{make}, \textbf{g++} pro C++ soubory a \textbf{gcc} pro C sobory. Volba překladače je všam možná pozměnit v přiloženém \texttt{Makefile}.


%---------------------------------------------------------------------------
\section{Použité zdroje}

Zde vypište, které zdroje jste použili k tvorbě: hotový kód, hotová data
(obrázky, modely, $\ldots$), studijní materiály. Pokud vyplyne, že v projektu
je použit kód nebo data, která nejsou uvedena tady, jedná se o závažný problém
a projekt bude pravděpodobně hodnocen 0 body.

Rozsah: potřebný počet odrážek

%---------------------------------------------------------------------------
\section{Nejdůležitější dosažené výsledky}

Popište 3 věci, které jsou na vašem projektu nejlepší. Nejlépe ukažte a
komentujte obrázky, v nejhorším případě vypište textově.

%---------------------------------------------------------------------------
\section{Použití programu}

\subsection{Překlad programu}
V kořenovém adresáři spusťte příkaz
\begin{itemize}
  \item[] \texttt{\$ make}
\end{itemize}
Provede se překlad programové části.


\subsection{Spuštění programu}
V kořenovém adrešáři použijte příkaz
\begin{itemize}
  \item[] \texttt{\$ make run-cpu}
\end{itemize}
nebo
\begin{itemize}
  \item[] \texttt{\$ make run-gpu}
\end{itemize}
pro spuštění výpočtu radiosity na cpu, resp. na gpu.


\subsection{Ovládání programu}
Po spuštění programu je zobrazena scéna. Scénou lze rotovat pomocí levého tlačítka myši a přibližovat ji pomocí pravého tlačítka myši.

Samotný výpočet radiosity probíhá po stisknutí klávesy \texttt{t}, a to podle parametru spuštění na procesoru, nebo grafické kartě.

Program lze ukončit pomocí standardních akcí (uzavření okna, apod.) nebo stisknutím klávesy \texttt{Escape}, \texttt{q} či \texttt{x}.


%---------------------------------------------------------------------------
\section{Zvláštní použité znalosti}

Uveďte informace, které byly potřeba nad rámec výuky probírané na FIT.
Vysvětlete je pomocí obrázků, schémat, vzorců apod. 

Rozsah: podle potřeby 

%---------------------------------------------------------------------------
\section{Rozdělení práce v týmu}

\begin{itemize}
\item Franta: udělal tohle, udělal tableto, ještě taky toto, vedl tým.
\item Pepa: pracoval na tom, na tomhle a ještě na tomto.
\item Mařenka: vytvořila tohle, tamto a ještě něco.
\end{itemize}
Pokud to bude vhodné, použijte odrážky místo souvislých vět.

Rozsah: co nejstručnější tak, aby bylo zřejmé, jak byla dělena práce a za co v
projektu je kdo zodpovědný.

%---------------------------------------------------------------------------
\section{Co bylo nejpracnější}

Popište, co vám při řešení nejvíce komplikovalo život, s čím jste se museli
potýkat, co zabralo čas.

Rozsah: 5-10 řádků

%---------------------------------------------------------------------------
\section{Zkušenosti získané řešením projektu}

Popište, co jste se řešením projektu naučili. Zahrňte dovednosti obecně
programátorské, věci z oblasti počítačové grafiky, ale i spolupráci v týmu,
hospodaření s časem, atd.

Rozsah: formulujte stručně, uchopte cca 3-5 věcí

%---------------------------------------------------------------------------
\section{Autoevaluace}

Ohodnoťte vaše řešení v jednotlivých kategoriích (0 – nic neuděláno,
zoufalství, 100\% – dokonalost sama). Projekt, který ve finále obdrží plný
počet bodů, může mít složky hodnocené i hodně nízko. Uvedení hodnot blízkých
100\% ve všech nebo mnoha kategoriích může ukazovat na nepochopení problematiky
nebo na snahu kamuflovat slabé stránky projektu. Bodově hodnocena bude i
schopnost vnímat silné a slabé stránky svého řešení.

\paragraph{Technický návrh (50\%):} (analýza, dekompozice problému, volba
vhodných prostředků, $\ldots$) 
Stručně (1-2 řádky) komentujte hodnocení. 

\paragraph{Programování (50\%):} (kvalita a čitelnost kódu, spolehlivost běhu,
obecnost řešení, znovupoužitelnost, $\ldots$)
Stručně (1-2 řádky) komentujte hodnocení. 

\paragraph{Vzhled vytvořeného řešení (50\%):} (uvěřitelnost zobrazení,
estetická kvalita, vhled GUI, $\ldots$)
Stručně (1-2 řádky) komentujte hodnocení. 

\paragraph{Využití zdrojů (50\%):} (využití existujícího kódu a dat, využití
literatury, $\ldots$)
Stručně (1-2 řádky) komentujte hodnocení. 

\paragraph{Hospodaření s časem (50\%):} (rovnoměrné dotažení částí projektu,
míra spěchu, chybějící části řešení, $\ldots$)
Stručně (1-2 řádky) komentujte hodnocení. 

\paragraph{Spolupráce v týmu (50\%):} (komunikace, dodržování dohod, vzájemné
spolehnutí, rovnoměrnost, $\ldots$)
Stručně (1-2 řádky) komentujte hodnocení. 

\paragraph{Celkový dojem (50\%):} (pracnost, získané dovednosti, užitečnost,
volba zadání, cokoliv, $\ldots$)
Stručně (5-10 řádků) komentujte hodnocení. 

%---------------------------------------------------------------------------
\section{Doporučení pro budoucí zadávání projektů}

Co vám vyhovovalo a co nevyhovovalo na organizaci projektů? Které prvky by měly
být zachovány, zesíleny, potlačeny, eliminovány?

%---------------------------------------------------------------------------
\section{Různé}

Ještě něco by v dokumentaci mělo být? Napište to sem! Podle potřeby i založte
novou kapitolu.

\end{document}
% vim:set ft=tex expandtab enc=utf8:
